%%%%%%%%%%%%%%%%%%%% author.tex %%%%%%%%%%%%%%%%%%%%%%%%%%%%%%%%%%%
%
% sample root file for your "contribution" to a proceedings volume
%
% Use this file as a template for your own input.
%
%%%%%%%%%%%%%%%% Springer %%%%%%%%%%%%%%%%%%%%%%%%%%%%%%%%%%


\documentclass{svproc}
\usepackage{marvosym}
%
% RECOMMENDED %%%%%%%%%%%%%%%%%%%%%%%%%%%%%%%%%%%%%%%%%%%%%%%%%%%
%
\usepackage{graphicx}

% to typeset URLs, URIs, and DOIs
\usepackage{url}
\usepackage{hyperref}
\def\UrlFont{\rmfamily}
\providecommand{\doi}[1]{doi:\discretionary{}{}{}#1}

\def\orcidID#1{\unskip$^{[#1]}$}
\def\letter{$^{\textrm{(\Letter)}}$}

%%%%%% DEBUG section %%%%%%
\newcommand*{\DEBUG}{}

\ifdefined\DEBUG
\usepackage[T2A]{fontenc}
\usepackage[utf8]{inputenc}
%\usepackage[russian]{babel}
\usepackage[usenames]{color}
%\usepackage{colortbl}
\newcommand{\FIXME}[1]{ % описание
	\colorbox{yellow}{#1}
}
\else
\newcommand{\FIXME}[1]{ % описание
}
\fi


\begin{document}
\mainmatter              % start of a contribution
%
\title{Topological approach to finding blackholes in directed networks}
%
\titlerunning{Topological black holes mining}  % abbreviated title (for running head)
%                                     also used for the TOC unless
%                                     \toctitle is used
%
\author{Denis Ivanov\inst{1} \and Alexander Semenov\inst{2}}
%
\authorrunning{Ivanov, Semenov} % abbreviated author list (for running head)
%
%%%% list of authors for the TOC (use if author list has to be modified)
\tocauthor{Denis Ivanov and Alexander Semenov}
%
\institute{Lomonosov Moscow State University, Moscow, Russian Federation \\
\email{mr.salixnew@gmail.com}
\and
JSC NICEVT, Moscow, Russian Federation\\
\email{semenov@nicevt.ru}
}

\maketitle              % typeset the title of the contribution

\begin{abstract}
In this paper we consider the task of finding so called Blackhole pattern in directed unweighed graphs.
Firstly, we analyse already existing algorithms and approaches. Secondly, we describe how structure of a graph
affects their efficiency. Then we introduce our approach to graph preprocessing. Finally, we describe topological sort based heuristic
and provide results of experimental comparison between our approach and previously developed algorithm.
% We would like to encourage you to list your keywords within
% the abstract section using the \keywords{...} command.
\keywords{Directed networks $\cdot$ Subgraph mining}
\end{abstract}
%
\section{Blackhole search in directed graph}
%

%
\subsection{Task statement}
%

%
\subsubsection{Subsection example}
%

%
\subsection{Related Work}
\cite{li2010detecting,li2012mining,li2014mining,hong2015detecting}

%
\subsection{Known Issues}
%

%
\section{Algorithm Design}
We propose to look at the blackhole search problem from the following: it would be divided into two independent tasks. 
The first one is graph preprocessing. Here we aim to simplify graph structure as far as it is possible with respect to a restricted time span. 
Decreased graph scale would definitely speed up the the calculation, as it can fall into a brute-force search in most cases.
The second is the blackhole search itself. This step is done after preprocessing and the goal here is to find as many blackholes as possible in a restricted time.
%

%
\subsection{Graph Preprocessing}
\FIXME{не забыть описать в KnownIssues}

\FIXME{не забыть вставить ссылки на RMAT \cite{chakrabarti2004r}, SSCA2 \cite{bader2005design}, \cite{random-uniform}, SmallWorld \cite{watts1999networks}}

As it was described in the Known Issues section, large SCCs waste lots of computational time without adding any new blackholes. 
For certain graph types (RMAT, SmallWorld) it can be crucial, because one SCC can contain up to 100\% of all graph's vertices. 
In such a case, it would be nice to consider large SCC as a single vertex, aggregating all incoming and outcoming peripheral edges
of original SCC.
On the preprocessing stage of the algorithm we reduce every SCC to a single vertex. The algorithm is not new and known as a graph condensation.

More formally:\par
\begin{definition}
TODO: Graph condensation definition
\end{definition}

In this article we use the following algorithm for finding graph condensation.
It was independently proposed by Kosaraju and Sharir back in 1979. \FIXME{Ссылку на эту статью} \par

First, we find topological sort of the initial graph. It takes O(E + V).
\begin{definition}
TODO: Topological sort
\end{definition}

\begin{definition}
Closure of vertex v is a set of all nodes reachable from v, including v.
\end{definition}

\begin{definition}
TODO: Reachable
\end{definition}

When we finally have an array, containing graph vertices in the topologically sorted order, we do the following:

Define global array of used vertices. Initially, all vertices are unused.
Then for every vertex from bottom to top:
\begin{enumerate}
    \item Acquire closure of unused vertices reachable from current vertex; 
    \item Mark every vertex in the closure as used;
    \item All the acquired vertices belong to the same SCC.
\end{enumerate}

The fact given in the last step can be shortly explained in the following way:
topological sort guarantees that any vertex has index larger than indeces of her children.
That is why, given that we process vertices from bottom to top, by the moment we process the vertex, all it's children are already marked used and no longer available for search.
As all the ways down are prohibited and there's no way up, we will acquire an SCC.\par

As a result of the algorithm, described above, we have each vertex associated with certain SCC.
SCCs itself form a graph with number of vertices less than or equal to one of the initial graph.

Finally, we need to remove edges duplicates and the graph condesation is built. 


%

%
\subsection{Blackholes search}
When the graph is prepared, we can move on to searching blackholes. The task is combinatorial in its nature, therefore, we aim to
decrease the number of potential black holes.
To understand the issue, we first describe BruteForce approach. \par

\begin{definition}
TODO: simple blackhole
\end{definition}

\begin{definition}
TODO: root of blackhole
\end{definition}

\begin{definition}
TODO: basis of blackhole
\end{definition}

\begin{definition}
TODO: complex blackhole
\end{definition}

\begin{definition}
TODO:
\end{definition}

To BruteForce the task we iterate over all possible unordered sets of nodes. 
Then for every node we acquire its closure. Union of all such closures is a candidate blackhole.
If candidate blackhole is a weakly connected subgraph, then it is a blackhole.

With BruteForce algorithm it is possible to find the same blackhole several times.
We aim to avoid at least some of the duplicates and introduce heuristic for it.

According to the definitions above if some node is not a root of blackhole, it can be omitted, as set of roots
uniquely identify a blackhole.
Every non-root node in blackhole is reachable from one of the roots. Therefore, if we know reachability matrix for
the given graph, we can skip bloated non-basis combinations of nodes.
Of course, we could straightforward build the reachability matrix. But, it takes O(V \^ 3) time. Even though we decreased graph size
in preprocessing step, it still can result in lots of computation. That is why, we will only try to partially skip duplicates and the rest will
be filtered later.\par

Let's consider a closure of a single blackhole root. Let this closure to be acquired by depth-first search, which, basically means, that
we can lay those nodes in topologically sorted order. Also, while traversing this graph, we can calculate size of closure for each node
in the given roots closure. Finally, if some node lays on the i'th position on the topological sort and its closure size is i, then we can 
conclude, that every node with index less than i in the topological sort will be reachable from v(i).

%

%
\section{Experimental part}
To demonstrate efficiency of our approach, we conducted a series of experiments. We ran both algorithms (New Jersey team and ours) on a 
set of graphs. \par
For these experiments we chose RMAT, Uniform Random (UR) and SSCA2 graph types. Those were generated for scales from 4 to 22
with step 2. Size i means that graph has 2 pow i nodes and 32 * V edges. Actual count of edges to process can differ from these ideal values, as duplicate edges 
and self-loops are omitted when reading graph from file. \par

\begin{table}[]
\caption{Configurations of experimental graphs}
\label{tabular:tableconfigurations}
\begin{center}
\begin{tabular}{ccc}
Graph Type & Nodes Count & Edges Count \\
RMAT.04 & 16 & 177 \\
RMAT.06 & 64 & 1270 \\
RMAT.08 & 256 & 6663 \\
RMAT.10 & 1024 & 30186 \\
RMAT.12 & 4096 & 126988 \\
RMAT.14 & 16384 & 517554 \\
RMAT.16 & 65536 & 2087072 \\
RMAT.18 & 262144 & 7892410 \\
RMAT.20 & 1048576 & 32993233 \\
RMAT.22 & 4194304 & 67099679 \\
SSCA2.04 & 16 & 120 \\
SSCA2.06 & 64 & 722 \\
SSCA2.08 & 256 & 2668 \\
SSCA2.10 & 1024 & 10529 \\
SSCA2.12 & 4096 & 43603 \\
SSCA2.14 & 16384 & 173477 \\
SSCA2.16 & 65536 & 677649 \\
SSCA2.18 & 262144 & 2724762 \\
SSCA2.20 & 1048576 & 10902287 \\
SSCA2.22 & 4194304 & 43580024 \\
UR.04 & 16 & 207 \\
UR.06 & 64 & 1614 \\
UR.08 & 256 & 7652 \\
UR.10 & 1024 & 32201 \\
UR.12 & 4096 & 130486 \\
UR.14 & 16384 & 523740 \\
UR.16 & 65536 & 2096570 \\
UR.18 & 262144 & 8388074 \\
UR.20 & 1048576 & 33553908 \\
UR.22 & 4194304 & 134217174
\end{tabular}
\end{center}
\end{table}

We've been conducting our experiments on a Linux machine with the following specifications:
\begin{itemize}
    \item OS: Ubuntu 16.04 xenial
    \item Kernel: x86\_64 Linux 4.15.0-46-generic
    \item Shell: bash 4.3.48
    \item CPU: Intel Core i7-8550U CPU @ 4GHz
    \item RAM: 15802MiB
\end{itemize}

All runs were performed in a single-thread mode. We allowed up to 20 minutes of working time for every run. In order to save time,
algorithms did not print out found blackholes, but printed counter on every found and not filtered (by any reason) blackhole.

\begin{table}[H]
\caption{Experimental results}
\label{tabular:tableresults}
\begin{center}
\begin{tabular}{ccccccccc}
Graph & Type & Algorithm & Blackholes & Time & Algorithm & Blackholes & Time \\
RMAT & 04 & NewJersey & 1 & 0.000265422 & TopSort & 1 & 0.000115912 \\
RMAT & 06 & NewJersey & 1 & 0.0121347 & TopSort & 1 & 0.00034812 \\
RMAT & 08 & NewJersey & 1 & 1.66685 & TopSort & 1 & 0.00146486 \\
RMAT & 10 & NewJersey & 6 & 1200 & TopSort & 10 & 0.00684045 \\
RMAT & 12 & NewJersey & 29 & 1200 & TopSort & 971738 & 1200 \\
RMAT & 14 & NewJersey & 187 & 1200 & TopSort & 735128 & 1200 \\
RMAT & 16 & NewJersey & 1192 & 1200 & TopSort & 373705 & 1200 \\
RMAT & 18 & NewJersey & 5176 & 1200 & TopSort & 10286 & 1200 \\
RMAT & 20 & NewJersey & 617 & 1200 & TopSort & 48867 & 1200 \\
RMAT & 22 & NewJersey & 98 & 1200 & TopSort & 0 & 1200 \\
SSCA2 & 04 & NewJersey & 16 & 0.154763 & TopSort & 16 & 0.000189925 \\
SSCA2 & 06 & NewJersey & 11 & 1200 & TopSort & 70 & 0.00121528 \\
SSCA2 & 08 & NewJersey & 42 & 1200 & TopSort & 263314 & 1200 \\
SSCA2 & 10 & NewJersey & 85 & 1200 & TopSort & 114411 & 1200 \\
SSCA2 & 12 & NewJersey & 343 & 1200 & TopSort & 5870 & 1200 \\
SSCA2 & 14 & NewJersey & 1433 & 1200 & TopSort & 323671 & 1200 \\
SSCA2 & 16 & NewJersey & 5829 & 1200 & TopSort & 893376 & 1200 \\
SSCA2 & 18 & NewJersey & 23179 & 1200 & TopSort & 224567 & 1200 \\
SSCA2 & 20 & NewJersey & 92347 & 1200 & TopSort & 24199 & 1200 \\
SSCA2 & 22 & NewJersey & 370770 & 1200 & TopSort & 0 & 1200 \\
UR & 04 & NewJersey & 1 & 0.000223768 & TopSort & 1 & 0.000115029 \\
UR & 06 & NewJersey & 1 & 0.0170713 & TopSort & 1 & 0.000331235 \\
UR & 08 & NewJersey & 1 & 2.30256 & TopSort & 1 & 0.00147483 \\
UR & 10 & NewJersey & 1 & 240.822 & TopSort & 1 & 0.00670724 \\
UR & 12 & NewJersey & 0 & 1200 & TopSort & 1 & 0.034011 \\
UR & 14 & NewJersey & 0 & 1200 & TopSort & 1 & 0.47257 \\
UR & 16 & NewJersey & 0 & 1200 & TopSort & 1 & 3.05003 \\
UR & 18 & NewJersey & 0 & 1200 & TopSort & 1 & 14.5788 \\
UR & 20 & NewJersey & 0 & 1200 & TopSort & 1 & 67.7176 \\
UR & 22 & NewJersey & 0 & 1200 & TopSort & 1 & 373.244
\end{tabular}
\end{center}
\end{table}

%

%
\section{Conclusion}
%

%
% ---- Bibliography ----
%
\bibliographystyle{spmpsci}
\bibliography{biblio}

\end{document}
