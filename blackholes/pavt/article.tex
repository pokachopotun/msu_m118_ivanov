%%%%%%%%%%%%%%%%%%%% author.tex %%%%%%%%%%%%%%%%%%%%%%%%%%%%%%%%%%%
%
% sample root file for your "contribution" to a proceedings volume
%
% Use this file as a template for your own input.
%
%%%%%%%%%%%%%%%% Springer %%%%%%%%%%%%%%%%%%%%%%%%%%%%%%%%%%


\documentclass{svproc}
\usepackage{marvosym}
%
% RECOMMENDED %%%%%%%%%%%%%%%%%%%%%%%%%%%%%%%%%%%%%%%%%%%%%%%%%%%
%
\usepackage{graphicx}

% to typeset URLs, URIs, and DOIs
\usepackage{url}
\usepackage{hyperref}
\def\UrlFont{\rmfamily}
\providecommand{\doi}[1]{doi:\discretionary{}{}{}#1}

\def\orcidID#1{\unskip$^{[#1]}$}
\def\letter{$^{\textrm{(\Letter)}}$}

%%%%%% DEBUG section %%%%%%
\newcommand*{\DEBUG}{}

\ifdefined\DEBUG
\usepackage[T2A]{fontenc}
\usepackage[utf8]{inputenc}
%\usepackage[russian]{babel}
\usepackage[usenames]{color}
%\usepackage{colortbl}
\newcommand{\FIXME}[1]{ % описание
	\colorbox{yellow}{#1}
}
\else
\newcommand{\FIXME}[1]{ % описание
}
\fi


\begin{document}
\mainmatter              % start of a contribution
%
\title{Topological approach to finding blackholes in directed networks}
%
\titlerunning{Topological black holes mining}  % abbreviated title (for running head)
%                                     also used for the TOC unless
%                                     \toctitle is used
%
\author{Denis Ivanov\inst{1} \and Alexander Semenov\inst{2}}
%
\authorrunning{Ivanov, Semenov} % abbreviated author list (for running head)
%
%%%% list of authors for the TOC (use if author list has to be modified)
\tocauthor{Denis Ivanov and Alexander Semenov}
%
\institute{Lomonosov Moscow State University, Moscow, Russian Federation \\
\email{mr.salixnew@gmail.com}
\and
JSC NICEVT, Moscow, Russian Federation\\
\email{semenov@nicevt.ru}
}

\maketitle              % typeset the title of the contribution

\begin{abstract}
In this paper we consider the task of finding so called Blackhole pattern in directed unweighed graphs.
Firstly, we analyse already existing algorithms and approaches. Secondly, we describe how structure of a graph
affects their efficiency. Then we introduce our approach to graph preprocessing. Finally, we describe topological sort based heuristic
and provide results of experimental comparison between our approach and previously developed algorithm.
% We would like to encourage you to list your keywords within
% the abstract section using the \keywords{...} command.
\keywords{Directed networks $\cdot$ Subgraph mining}
\end{abstract}
%
\section{Blackhole search in directed graph}
%

%
\subsection{Task statement}
%

%
\subsubsection{Subsection example}
%

%
\subsection{Related Work}
\cite{li2010detecting,li2012mining,li2014mining,hong2015detecting}

%
\subsection{Known Issues}
%

%
\section{Algorithm Design}
We propose to look at the blackhole search problem from the following: it would be divided into two independent tasks. 
The first one is graph preprocessing. Here we aim to simplify graph structure as far as it is possible with respect to a restricted time span. 
Decreased graph scale would definitely speed up the the calculation, as it can fall into a brute-force search in most cases.
The second is the blackhole search itself. This step is done after preprocessing and the goal here is to find as many blackholes as possible in a restricted time.
%

%
\subsection{Graph Preprocessing}
\FIXME{не забыть описать в KnownIssues}

\FIXME{не забыть вставить ссылки на RMAT \cite{chakrabarti2004r}, SSCA2 \cite{bader2005design}, \cite{random-uniform}, SmallWorld \cite{watts1999networks}}

\FIXME{не понял логику абзаца: большая SCC может жрать много времени, граф может из нее состоять, и лучше сжать ее до 1 вершины}

\FIXME{Хорошо бы ссылку на graph condensation}

As it was described in the Known Issues section, large SCCs waste lots of computational time without adding any new blackholes. 
For certain graph types (RMAT, SmallWorld) it can be crucial, because one SCC can contain up to 100\% of all graph's vertices. 
In such a case, it would be nice to consider large SCC as a single vertex, aggregating all incoming and outcoming peripheral edges
of original SCC.
On the preprocessing stage of the algorithm we reduce every SCC to a single vertex. The algorithm is not new and known as a graph condensation.

More formally:\par
\begin{definition}
TODO: Graph condensation definition
\end{definition}

In this article we use to use the following algorithm for finding graph condensation.\par

First, we find topological sort of the initial graph. It takes O(E + V).
\begin{definition}
TODO: Topolorgical sort
\end{definition}

\begin{definition}
TODO: Closure
\end{definition}

\begin{definition}
TODO: Reachable
\end{definition}

When we finally have an array, containing graph vertices in the topologically sorted order, we do the following:

Define global array of used vertices. Initially, all vertices are unused.
Then for every vertex from bottom to top:
\begin{enumerate}
    \item Acquire closure of unused vertices reachable from current vertex; 
    \item Mark every vertex in the closure as used;
    \item All the acquired vertices belong to the same SCC.
\end{enumerate}

The fact given in the last step can be shortly explained in the following way:
topological sort guarantees that any vertex has index larger than indeces of her children.
That is why, given that we process vertices from bottom to top, by the moment we process the vertex, all it's children are already marked used and no longer available for search.
As all the ways down are prohibited and there's no way up, we will acquire an SCC.\par

As a result of the algorithm, described above, we have each vertex associated with certain SCC.
SCCs itself form a graph with number of vertices less than or equal to one of the initial graph.

Finally, we need to remove edges duplicates and the graph condesation is built.

%

%
\subsection{Graph topology}
%

%
\section{Experimental part}
%

%
\section{Conclusion}
%

%
% ---- Bibliography ----
%
\bibliographystyle{spmpsci}
\bibliography{biblio}


%\begin{thebibliography}{6}
%\bibitem {a1:a2}
%Example
%\doi{doi:number}

%\bibitem {onlyurl}
%\NCBI: National Center for Biotechnology Information. \url{http://www.ncbi.nlm.nih.gov}

%\end{thebibliography}
\end{document}
