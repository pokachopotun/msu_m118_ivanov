%%%%%%%%%%%%%%%%%%%% author.tex %%%%%%%%%%%%%%%%%%%%%%%%%%%%%%%%%%%
%
% sample root file for your "contribution" to a proceedings volume
%
% Use this file as a template for your own input.
%
%%%%%%%%%%%%%%%% Springer %%%%%%%%%%%%%%%%%%%%%%%%%%%%%%%%%%


\documentclass{svproc}
\usepackage{marvosym}
%
% RECOMMENDED %%%%%%%%%%%%%%%%%%%%%%%%%%%%%%%%%%%%%%%%%%%%%%%%%%%
%
\usepackage{graphicx}

% to typeset URLs, URIs, and DOIs
\usepackage{url}
\usepackage{hyperref}
\def\UrlFont{\rmfamily}
\providecommand{\doi}[1]{doi:\discretionary{}{}{}#1}

\def\orcidID#1{\unskip$^{[#1]}$}
\def\letter{$^{\textrm{(\Letter)}}$}

\begin{document}
\mainmatter              % start of a contribution
%
\title{Topological approach to finding blackholes in directed networks}
%
\titlerunning{Topological black holes mining}  % abbreviated title (for running head)
%                                     also used for the TOC unless
%                                     \toctitle is used
%
\author{Denis Ivanov\inst{1} \and Alexander Semenov\inst{2}}
%
\authorrunning{Ivanov, Semenov} % abbreviated author list (for running head)
%
%%%% list of authors for the TOC (use if author list has to be modified)
\tocauthor{Denis Ivanov and Alexander Semenov}
%
\institute{Lomonosov Moscow State University, Moscow, Russian Federation \\
\email{mr.salixnew@gmail.com}
\and
JSC NICEVT, Moscow, Russian Federation\\
\email{semenov@nicevt.ru}
}

\maketitle              % typeset the title of the contribution

\begin{abstract}
In this paper we consider the task of finding so called Blackhole pattern in directed unweighed graphs.
Firstly, we analyse already existing algorithms and approaches. Secondly, we describe how structure of a graph
affects their efficiency. Then we introduce our approach to graph preprocessing. Finally, we describe topological sort based heuristic
and provide results of experimental comparison between our approach and previously developed algorithm.
% We would like to encourage you to list your keywords within
% the abstract section using the \keywords{...} command.
\keywords{Directed networks $\cdot$ Subgraph mining}
\end{abstract}
%
\section{Blackhole search in directed graph}
%

%
\subsection{Task statement}
%

%
\subsubsection{Subsection example}
%

%
\subsection{Related work}
\cite{li2010detecting,li2012mining,li2014mining,hong2015detecting}

%
\subsection{Known issues}
%

%
\section{Algorithm design}
We propose to look at the black hole search task from the following perspective: it would be divided into two independent 
tasks. 
The first one is graph preprocessing. Here we aim to simplify graph structure as far as it is possible with respect to
restricted time span. Decreased graph scale would definitely speed up the rest of the calculation, as it is result in brute force search in most cases.
The second is the blackhole search itself. This step is done after preprocessing and the goal here is to find as many blackholes as possible in restricted time.
%

%
\subsection{Graph perprocessing}
%

%
\subsection{Graph topology}
%

%
\section{Experimental part}
%

%
\section{Conclusion}
%

%
% ---- Bibliography ----
%
\bibliographystyle{spmpsci}
\bibliography{biblio}


%\begin{thebibliography}{6}
%\bibitem {a1:a2}
%Example
%\doi{doi:number}

%\bibitem {onlyurl}
%\NCBI: National Center for Biotechnology Information. \url{http://www.ncbi.nlm.nih.gov}

%\end{thebibliography}
\end{document}
