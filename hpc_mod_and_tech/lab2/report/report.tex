\documentclass[12pt,a4paper]{scrartcl}
\usepackage[utf8]{inputenc}
\usepackage[english,russian]{babel}

\usepackage{graphicx}
\usepackage{amsmath}
\usepackage{physics}

\begin{document}

\begin{titlepage}

    \begin{figure}[h]
        \centering
        \includegraphics[scale=0.5]{gzlogo_new.png}
    \end{figure}

    \begin{center}
        \large
        Московский государственный университет имени М.В. Ломоносова

        \vfill

        \LARGE
        \textbf{Задание 2. Численное решение параболическоого уравнения}

        \vfill

        \hfill\begin{minipage}{0.6\textwidth}
            \normalsize
            \textbf{Факультет:} Вычислительной математики и кабернетики \\
            \textbf{Кафедра:} Суперкомпьютеров и квантовой информатики \\
            \textbf{Группа:} М-218 \\
            \textbf{Студент:} Иванов Денис Евгеньевич \\
            \textbf{Предмет:} Суперкомпьютерное Моделирование и Технологии \\
        \end{minipage}
        \vfill
        

    \end{center}

    \begin{center}
        \large
        Москва, 2019
    \end{center}
\end{titlepage}

\Large
Математическая постановка дифференциальной задачи

\normalsize
В двумерной замкнутой области:

\begin{equation}
    \Omega = [0 \leq x \leq L_x] \times [0 \leq y \leq L_y]
\end{equation}

Для $0 < t \leq T$ требуется найти решение $u(x,y,t)$ равнения в частных производных

\begin{equation}
    \label{eq:derivative}
    \frac{\partial u}{\partial t} = \Delta u + \nabla div u
\end{equation}

Заметим, что функция $u$ - это вектор функция $u=(u_1, u_2)$ поэтому уравнения ()
можно записать в виде

\begin{equation}
    \begin{cases} 
    \label{eq:decompose}
        \frac{\partial{u_1}}{\partial t} = 2 * \pdv[2]{u_1}{x} + \pdv[2]{u_1}{y} + \pdv{u_2}{x}{y} \\ 
        \frac{\partial{u_2}}{\partial t} = \pdv[2]{u_2}{x} + 2 * \pdv[2]{u_2}{y} + \pdv{u_1}{x}{y} \\ 
    \end{cases}
\end{equation}



Дополним систему (\ref{eq:derivative}) начальными условиями

\begin{equation}
    u|_{t = 0} = \phi(x, y)
\end{equation}

На границах области заданы граничные условия первого рода, второго рода или периодические
\newpage

\Large
Численный метод решения задачи


\normalsize
Для численного решения задачи введем на области $\Omega$ равномерную решетку
$\Omega_h = \bar{\omega_h} \times \omega_{\tau}$, где 

\begin{equation}
\begin{array}{lcl}
    \bar{\omega_{h}} & = & \{u(x_i = ih_x, jh_y),i,j = 0,1,...,N, h_xN = L_x, h_y N = L_y\} \\
    \omega_{\tau} & = & \{t_n = n\tau, n = 0,1,...,K, \tau K = T\} \\
\end{array}
\end{equation}

 - множество внутренних узлов
 - множество граничных узлов сетки.

    Для аппроксимации исходного уравнения (1) с однородными граничными условиями и начальными условиями (2) воспользуемся следующей системой уравнений



Здесь через .... обозначен дискретный аналог дифференциального решения ... . А запись
... означает значение численного решения в точке ... .

 - пятиточечный разностный аналог оператора Лапласа:



... - девятиточечный аналог оператора ... :


приведенная выше разностная схема является явной - значение ... на ... шаге по времени явным образом выразить через значения на предыдущих слоях

Для параболических задач такое соотношение имеет вид .... , где ... - константа, которая вычисляется аналитически или подбирается экспериментально.

Для начала счета используем условие (2)


 В случае однородных граничных условий первого рода:

В случае однородных условий второго рода, используя более точную аппроксимацию (пример производной по переменной ... )




Разностная аппроксимация для периодических граничных условий выглядит следующим образом:
 

      
\end{document}
