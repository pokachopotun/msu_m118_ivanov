\documentclass[12pt,a4paper]{scrartcl}
\usepackage[utf8]{inputenc}
\usepackage[english,russian]{babel}

\usepackage{graphicx}
\usepackage{amsmath}
\usepackage{physics}

\begin{document}

\begin{titlepage}

    \begin{figure}[h]
        \centering
        \includegraphics[scale=0.5]{gzlogo_new.png}
    \end{figure}

    \begin{center}
        \large
        Московский государственный университет имени М.В. Ломоносова

        \vfill

        \LARGE
        \textbf{Задание 2. Численное решение параболическоого уравнения}

        \vfill

        \hfill\begin{minipage}{0.6\textwidth}
            \normalsize
            \textbf{Факультет:} Вычислительной математики и кабернетики \\
            \textbf{Кафедра:} Суперкомпьютеров и квантовой информатики \\
            \textbf{Группа:} М-218 \\
            \textbf{Студент:} Иванов Денис Евгеньевич \\
            \textbf{Предмет:} Суперкомпьютерное Моделирование и Технологии \\
        \end{minipage}
        \vfill
        

    \end{center}

    \begin{center}
        \large
        Москва, 2019
    \end{center}
\end{titlepage}

\section{Математическая постановка дифференциальной задачи}

В двумерной замкнутой области:

\begin{equation}
    \Omega = [0 \leq x \leq L_x] \times [0 \leq y \leq L_y]
\end{equation}

Для $0 < t \leq T$ требуется найти решение $u(x,y,t)$ равнения в частных производных

\begin{equation}
    \label{eq:derivative}
    \frac{\partial u}{\partial t} = \Delta u + \nabla div u
\end{equation}

Заметим, что функция $u$ - это вектор функция $u=(u_1, u_2)$ поэтому уравнения (\ref{eq:derivative})
можно записать в виде

\begin{equation}
    \begin{cases} 
    \label{eq:decompose}
        \frac{\partial{u_1}}{\partial t} = 2 * \pdv[2]{u_1}{x} + \pdv[2]{u_1}{y} + \pdv{u_2}{x}{y} \\ 
        \frac{\partial{u_2}}{\partial t} = \pdv[2]{u_2}{x} + 2 * \pdv[2]{u_2}{y} + \pdv{u_1}{x}{y} \\ 
    \end{cases}
\end{equation}

Дополним систему (\ref{eq:decompose}) начальными условиями

\begin{equation}
    \label{eq:initialcondition}
    u|_{t = 0} = \phi(x, y)
\end{equation}

На границах области заданы граничные условия первого рода, второго рода или периодические
\newpage

\section{Численный метод решения задачи}

Для численного решения задачи введем на области $\Omega$ равномерную решетку
$\Omega_h = \bar{\omega_h} \times \omega_{\tau}$, где 

\begin{equation}
\begin{array}{lcl}
    \bar{\omega_{h}} & = & \{u(x_i = ih_x, y_j = jh_y),i,j = 0,1,...,N, h_xN = L_x, h_y N = L_y\} \\
    \omega_{\tau} & = & \{t_n = n\tau, n = 0,1,...,K, \tau K = T\} \\
\end{array}
\end{equation}

$\omega_h$ - множество внутренних узлов

$\omega_{\gamma}$ - множество граничных узлов сетки.

    Для аппроксимации исходного уравнения (\ref{eq:decompose}) с однородными граничными условиями и начальными условиями (\ref{eq:initialcondition}) воспользуемся следующей системой уравнений

\begin{equation}
    \label{eq:system}
        \frac{v_{ij}^{n+1} + v_{ij}^n}{\tau} = \Delta^hv^n + \nabla^h{div}^hv^n, (x_i, y_j) \in \omega_h, n = 1,2,...,K-1
\end{equation}

Здесь через $v = (v_1, v_2)$ обозначен дискретный аналог дифференциального решения $u = (u_1, u_2)$ . А запись
$v_{ij}^n$ означает значение численного решения в точке $(x_i,y_j,t_n) \in \bar{\omega_h} \times \omega_{\tau}$.

$\Delta^h$ - пятиточечный разностный аналог оператора Лапласа:

\begin{equation}
    \label{eq:laplas}
        \Delta^hv^n = \frac{v_{i-1j}^n - 2v_{ij}^n + v_{i+1j}^n}{h_x^2} + \frac{v_{ij-1}^n - 2v_{ij}^n + v_{ij+1}^n}{h_y^2}
\end{equation}



$\nabla^hdiv^h$ - девятиточечный аналог оператора $\nabla div$:
\begin{equation}
    \label{eq:nabladiv}
        \nabla^hdiv^hv^n  = 
        \left(
            \begin{array}{lcr} 
                \frac{v1_{i-1j}^n - 2v1_{ij}^n + v1_{i+1j}^n}{h_x^2} & + & \frac{v2_{i+1j+1}^n - v2_{i+1j-1}^n - v2_{i-1j+1}^n + v2_{i-1j-1}^n}{4h_xh_y} \\
                \frac{v2_{ij-1}^n - 2v2_{ij}^n + v2_{ij+1}^n}{h_y^2} & + & \frac{v2_{i+1j+1}^n - v2_{i+1j-1}^n - v2_{i-1j+1}^n + v2_{i-1j-1}^n}{4h_xh_y} 
     \end{array}\right)
\end{equation}

приведенная выше разностная схема является явной - значение $v_{ij}^{n+1}$ на $(n+1)$ шаге по времени можно явным образом выразить через значения на предыдущих слоях

Для параболических задач такое соотношение имеет вид $\tau < \gamma(h_x^2, h_y^2)$ , где $\gamma$ - константа, которая вычисляется аналитически или подбирается экспериментально.

Для начала счета используем условие (\ref{eq:initialcondition}):

\begin{equation}
    \label{eq:begincalc}
        v_{ij}^0 = \phi(x_i,y_j), (x_i,y_j) \in \omega_h
\end{equation}

В случае однородных граничных условий первого рода:

\begin{equation}
    \label{eq:secondtype}
        v_{ij}^n = 0, (x_i,y_j) \in \gamma_h
\end{equation}

В случае однородных условий второго рода, используем более точную аппроксимацию (пример производной по переменной $x$):


\begin{equation}
    \label{eq:consistentsecond}
    \begin{array}{lcl}
        \frac{\partial v}{\partial x}|_{x = 0} & \approx & \frac{-3v_{0j} + 4v_{1j} - v_{2j}}{2h_x}, (x_i, y_j) \in \omega_h \\
        \frac{\partial v}{\partial x}|_{x = L_x} & \approx & \frac{v_{N-2j} - 4v{N-1j} + 3v_{Nj}}{2h_x}, (x_i, y_j) \in \omega_h \\
    \end{array}
\end{equation}

Разностная аппроксимация для периодических граничных условий выглядит следующим образом:

\begin{equation}
    \label{eq:periodicapprox}
    \begin{array}{lcl}
        v_{0j}^{n+1} = v_{Nj}^{n+1} & , & v_{1j}^{n+1} = v_{N+1j}^{n+1},j=0,1,...,N \\
        v_{i0}^{n+1} = v_{iN}^{n+1} & , & v_{i1}^{n+1} = v_{iN+1}^{n+1},j=0,1,...,N \\
    \end{array}
\end{equation}
      
\end{document}
