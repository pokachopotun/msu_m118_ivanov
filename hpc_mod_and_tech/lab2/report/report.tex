\documentclass[12pt,a4paper]{scrartcl}
\usepackage[utf8]{inputenc}
\usepackage[english,russian]{babel}

\usepackage{graphicx}
\usepackage{amsmath}
\usepackage{physics}

\begin{document}

\begin{titlepage}

    \begin{figure}[h]
        \centering
        \includegraphics[scale=0.5]{gzlogo_new.png}
    \end{figure}

    \begin{center}
        \large
        Московский государственный университет имени М.В. Ломоносова

        \vfill

        \LARGE
        \textbf{Задание 2. Численное решение параболическоого уравнения}

        \vfill

        \hfill\begin{minipage}{0.6\textwidth}
            \normalsize
            \textbf{Факультет:} Вычислительной математики и кабернетики \\
            \textbf{Кафедра:} Суперкомпьютеров и квантовой информатики \\
            \textbf{Группа:} М-218 \\
            \textbf{Студент:} Иванов Денис Евгеньевич \\
            \textbf{Предмет:} Суперкомпьютерное Моделирование и Технологии \\
        \end{minipage}
        \vfill

    \end{center}

    \begin{center}
        \large
        Москва, 2019
    \end{center}
\end{titlepage}

\section{Математическая постановка дифференциальной задачи}

В двумерной замкнутой области:

\begin{equation}
    \Omega = [0 \leq x \leq L_x] \times [0 \leq y \leq L_y]
\end{equation}

Для $0 < t \leq T$ требуется найти решение $u(x,y,t)$ равнения в частных производных

\begin{equation}
    \label{eq:derivative}
    \frac{\partial u}{\partial t} = \Delta u + \nabla div u
\end{equation}

Заметим, что функция $u$ - это вектор функция $u=(u_1, u_2)$ поэтому уравнения (\ref{eq:derivative})
можно записать в виде

\begin{equation}
    \begin{cases} 
    \label{eq:decompose}
        \frac{\partial{u_1}}{\partial t} = 2 * \pdv[2]{u_1}{x} + \pdv[2]{u_1}{y} + \pdv{u_2}{x}{y} \\ 
        \frac{\partial{u_2}}{\partial t} = \pdv[2]{u_2}{x} + 2 * \pdv[2]{u_2}{y} + \pdv{u_1}{x}{y} \\ 
    \end{cases}
\end{equation}

Дополним систему (\ref{eq:decompose}) начальными условиями

\begin{equation}
    \label{eq:initialcondition}
    u|_{t = 0} = \phi(x, y)
\end{equation}

На границах области заданы граничные условия первого рода, второго рода или периодические
\newpage

\section{Численный метод решения задачи}

Для численного решения задачи введем на области $\Omega$ равномерную решетку
$\Omega_h = \bar{\omega_h} \times \omega_{\tau}$, где

\begin{equation}
\begin{array}{lcl}
    \bar{\omega_{h}} & = & \{u(x_i = ih_x, y_j = jh_y),i,j = 0,1,...,N, h_xN = L_x, h_y N = L_y\} \\
    \omega_{\tau} & = & \{t_n = n\tau, n = 0,1,...,K, \tau K = T\} \\
\end{array}
\end{equation}

$\omega_h$ - множество внутренних узлов

$\omega_{\gamma}$ - множество граничных узлов сетки.

    Для аппроксимации исходного уравнения (\ref{eq:decompose}) с однородными граничными условиями и начальными условиями (\ref{eq:initialcondition}) воспользуемся следующей системой уравнений

\begin{equation}
    \label{eq:system}
        \frac{v_{ij}^{n+1} + v_{ij}^n}{\tau} = \Delta^hv^n + \nabla^h{div}^hv^n, (x_i, y_j) \in \omega_h, n = 1,2,...,K-1
\end{equation}

Здесь через $v = (v_1, v_2)$ обозначен дискретный аналог дифференциального решения $u = (u_1, u_2)$ . А запись
$v_{ij}^n$ означает значение численного решения в точке $(x_i,y_j,t_n) \in \bar{\omega_h} \times \omega_{\tau}$.

$\Delta^h$ - пятиточечный разностный аналог оператора Лапласа:

\begin{equation}
    \label{eq:laplas}
        \Delta^hv^n = \frac{v_{i-1j}^n - 2v_{ij}^n + v_{i+1j}^n}{h_x^2} + \frac{v_{ij-1}^n - 2v_{ij}^n + v_{ij+1}^n}{h_y^2}
\end{equation}



$\nabla^hdiv^h$ - девятиточечный аналог оператора $\nabla div$:
\begin{equation}
    \label{eq:nabladiv}
        \nabla^hdiv^hv^n  = 
        \left(
            \begin{array}{lcr} 
                \frac{v1_{i-1j}^n - 2v1_{ij}^n + v1_{i+1j}^n}{h_x^2} & + & \frac{v2_{i+1j+1}^n - v2_{i+1j-1}^n - v2_{i-1j+1}^n + v2_{i-1j-1}^n}{4h_xh_y} \\
                \frac{v2_{ij-1}^n - 2v2_{ij}^n + v2_{ij+1}^n}{h_y^2} & + & \frac{v2_{i+1j+1}^n - v2_{i+1j-1}^n - v2_{i-1j+1}^n + v2_{i-1j-1}^n}{4h_xh_y} 
     \end{array}\right)
\end{equation}

приведенная выше разностная схема является явной - значение $v_{ij}^{n+1}$ на $(n+1)$ шаге по времени можно явным образом выразить через значения на предыдущих слоях

Для параболических задач такое соотношение имеет вид $\tau < \gamma(h_x^2, h_y^2)$ , где $\gamma$ - константа, которая вычисляется аналитически или подбирается экспериментально.

Для начала счета используем условие (\ref{eq:initialcondition}):

\begin{equation}
    \label{eq:begincalc}
        v_{ij}^0 = \phi(x_i,y_j), (x_i,y_j) \in \omega_h
\end{equation}

В случае однородных граничных условий первого рода:

\begin{equation}
    \label{eq:secondtype}
        v_{ij}^n = 0, (x_i,y_j) \in \gamma_h
\end{equation}

В случае однородных условий второго рода, используем более точную аппроксимацию (пример производной по переменной $x$):


\begin{equation}
    \label{eq:consistentsecond}
    \begin{array}{lcl}
        \frac{\partial v}{\partial x}|_{x = 0} & \approx & \frac{-3v_{0j} + 4v_{1j} - v_{2j}}{2h_x}, (x_i, y_j) \in \omega_h \\
        \frac{\partial v}{\partial x}|_{x = L_x} & \approx & \frac{v_{N-2j} - 4v{N-1j} + 3v_{Nj}}{2h_x}, (x_i, y_j) \in \omega_h \\
    \end{array}
\end{equation}

Разностная аппроксимация для периодических граничных условий выглядит следующим образом:

\begin{equation}
    \label{eq:periodicapprox}
    \begin{array}{lcl}
        v_{0j}^{n+1} = v_{Nj}^{n+1} & , & v_{1j}^{n+1} = v_{N+1j}^{n+1},j=0,1,...,N \\
        v_{i0}^{n+1} = v_{iN}^{n+1} & , & v_{i1}^{n+1} = v_{iN+1}^{n+1},j=0,1,...,N \\
    \end{array}
\end{equation}

\section{Описание реализации}

Решение задачи было реализовано на языке С++ с использованием библиотек MPI и OpenMP.

Узлы суперкомпьютера расположены в порядке прямоугольной сетки. Эта сетка накладывается на расчетную сетку,
и каждый узел получает соответствующий участок. Если по какой-то из координат $x,y$ расчетная сетка не делится нацело
на количество узлов, то остаток отдается последнему узлу по данной координате. Это создает незначительный дисбаланс нагрузки.

Каждый узел хранит в памяти свой расчетный участок и дополнительный слой элементов толщиной 1, т.н. гало. Гало - пограничные элементы,
рассчитывается соседними узлами и передается на текущий.
Таким образом, у данного узла есть достаточно данных, чтобы рассчитать свою область на текущей итерации.

Для реализации циклических условий, расчетная область расширяется таким образом, чтобы можно было вычислить крайний элемент $N$.
Далее этот элемент пересылается на противоположную сторону сетки и помещается в индекс $0$. И, наоборот, первый элемент сетки
пересылается, чтобы расположиться в $N+1$ позиции. Т.к. крайнему узлу не требуется гало с той стороны, где сетка заканчивается,
то гало ячейки используются для расширения сетки.

На Bluegene код выполнялся с использованием OpenMP, в один, два или три потока. 
Запуск в один OpenMP поток считаем равносильным запуску без OpenMP. Производительность различных конфигураций
считалась относительно запусков на 128 узлах в один поток. Для сборки кода на bluegene была использована следующая команда:

\begin{center}
\small
mpixlcxxi\_r -O2 main.cpp -o bluegene -qsmp=omp
\end{center}

На Polus запуск с одной OpenMP нитью почти не показывал роста производительности
при увеличении числа узлов. Однако, если убрать OpenMP директивы из кода программы и скомпилировать без оптимизаций,
то прирост производительности становится очевиден. Для сборки кода на polus была использована следующая команда:

\begin{center}
\small
mpixlC -O0 main.cpp -o polus
\end{center}

Программа запускается командой

\begin{center}
\small
mpirun -n X ./solution mpiN mpiM gridN gridM gridT xmax ymax ompNumThreads
\end{center}

Где:
\begin{itemize}
    \item X - число MPI узлов
    \item mpiN, mpiM - размер сетки MPI
    \item gridN, gridM - размер вычислительной сетки
    \item gridT - количество шагов по времени
    \item xmax, ymax - ограничения по $x, y$
    \item ompNumThreads - число потоков OpenMP (отсутствует для Polus)
\end{itemize}

\newpage

\section{Результаты}

\begin{center}
\begin{table}[h]
\begin{tabular}{l|lllllll}
\label{tabular:bluegene}
HPC & Nodes & Grid & OMP & Wall time & Delta & Speedup & Efficiency \\
\hline
bluegene & 128 & 8192 & 1 & 4.96e+00 & 8.24e-08 & 1.00e+00 & 1.00e+00 \\
bluegene & 128 & 8192 & 2 & 2.49e+00 & 8.24e-08 & 1.99e+00 & 9.94e-01 \\
bluegene & 128 & 8192 & 3 & 1.67e+00 & 8.24e-08 & 2.96e+00 & 9.87e-01 \\
bluegene & 128 & 16384 & 1 & 2.09e+01 & 2.06e-08 & 1.00e+00 & 1.00e+00 \\
bluegene & 128 & 16384 & 2 & 1.05e+01 & 2.06e-08 & 2.00e+00 & 9.98e-01 \\
bluegene & 128 & 16384 & 3 & 6.98e+00 & 2.06e-08 & 2.99e+00 & 9.98e-01 \\
bluegene & 128 & 32768 & 1 & 8.32e+01 & 5.15e-09 & 1.00e+00 & 1.00e+00 \\
bluegene & 128 & 32768 & 2 & 4.17e+01 & 5.15e-09 & 1.99e+00 & 9.97e-01 \\
bluegene & 128 & 32768 & 3 & 2.79e+01 & 5.15e-09 & 2.99e+00 & 9.95e-01 \\
bluegene & 256 & 8192 & 1 & 2.48e+00 & 8.24e-08 & 2.00e+00 & 9.98e-01 \\
bluegene & 256 & 8192 & 2 & 1.25e+00 & 8.24e-08 & 3.95e+00 & 9.88e-01 \\
bluegene & 256 & 8192 & 3 & 8.45e-01 & 8.24e-08 & 5.86e+00 & 9.77e-01 \\
bluegene & 256 & 16384 & 1 & 1.04e+01 & 2.06e-08 & 2.00e+00 & 9.99e-01 \\
bluegene & 256 & 16384 & 2 & 5.23e+00 & 2.06e-08 & 3.99e+00 & 9.97e-01 \\
bluegene & 256 & 16384 & 3 & 3.50e+00 & 2.06e-08 & 5.96e+00 & 9.94e-01 \\
bluegene & 256 & 32768 & 1 & 4.17e+01 & 5.15e-09 & 2.00e+00 & 9.98e-01 \\
bluegene & 256 & 32768 & 2 & 2.12e+01 & 5.15e-09 & 3.93e+00 & 9.83e-01 \\
bluegene & 256 & 32768 & 3 & 1.42e+01 & 5.15e-09 & 5.88e+00 & 9.80e-01 \\
bluegene & 512 & 8192 & 1 & 1.24e+00 & 8.24e-08 & 3.99e+00 & 9.96e-01 \\
bluegene & 512 & 8192 & 2 & 6.36e-01 & 8.24e-08 & 7.80e+00 & 9.75e-01 \\
bluegene & 512 & 8192 & 3 & 4.33e-01 & 8.24e-08 & 1.14e+01 & 9.54e-01 \\
bluegene & 512 & 16384 & 1 & 5.22e+00 & 2.06e-08 & 4.00e+00 & 1.00e+00 \\
bluegene & 512 & 16384 & 2 & 2.63e+00 & 2.06e-08 & 7.94e+00 & 9.93e-01 \\
bluegene & 512 & 16384 & 3 & 1.76e+00 & 2.06e-08 & 1.19e+01 & 9.88e-01 \\
bluegene & 512 & 32768 & 1 & 2.08e+01 & 5.15e-09 & 3.99e+00 & 9.98e-01 \\
bluegene & 512 & 32768 & 2 & 1.05e+01 & 5.15e-09 & 7.92e+00 & 9.90e-01 \\
bluegene & 512 & 32768 & 3 & 7.03e+00 & 5.15e-09 & 1.18e+01 & 9.87e-01 \\
\hline
\end{tabular}
\caption{Результаты запусков на Bluegene.}
\end{table}
\end{center}

\newpage
\begin{center}
\begin{table}[h]
\begin{tabular}{l|lllllll}
\label{tabular:polus}
HPC & Nodes & Grid & Wall time & Delta & Speedup & Efficiency \\
\hline
polus & 1 & 2048 & 1.48e+01 & 1.32e-06 & 1.00e+00 & 1.00e+00 \\
polus & 1 & 4096 & 5.60e+01 & 3.30e-07 & 1.00e+00 & 1.00e+00 \\
polus & 1 & 8192 & 2.20e+02 & 8.24e-08 & 1.00e+00 & 1.00e+00 \\
polus & 10 & 2048 & 1.57e+00 & 1.32e-06 & 9.41e+00 & 9.41e-01 \\
polus & 10 & 4096 & 5.94e+00 & 3.30e-07 & 9.43e+00 & 9.43e-01 \\
polus & 10 & 8192 & 2.34e+01 & 8.24e-08 & 9.41e+00 & 9.41e-01 \\
polus & 20 & 2048 & 1.09e+00 & 1.32e-06 & 1.35e+01 & 6.77e-01 \\
polus & 20 & 4096 & 3.58e+00 & 3.30e-07 & 1.56e+01 & 7.82e-01 \\
polus & 20 & 8192 & 1.38e+01 & 8.24e-08 & 1.60e+01 & 8.00e-01 \\
polus & 40 & 2048 & 1 & IN QUEUE \\
polus & 40 & 4096 & 1 & IN QUEUE \\
polus & 40 & 8192 & 1 & IN QUEUE \\
\hline

\end{tabular}
\caption{Результаты запусков на Polus}
\end{table}
\end{center}

\end{document}
