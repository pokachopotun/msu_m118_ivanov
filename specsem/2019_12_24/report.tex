\documentclass[12pt,a4paper]{scrartcl}
\usepackage[utf8]{inputenc}
\usepackage[english,russian]{babel}

\usepackage{graphicx}
\usepackage{amsmath}

\begin{document}

\Large

\begin{center}
\textbf{Алгоритмы поиска подграфов типа <<черная дыра>> в ориентированном графе}\\
	Осенний семестр 2019 года \\
Краткий отчет по НИР 
\end{center}

\hfill\begin{minipage}{0.6\textwidth}
    \normalsize
    \textbf{Факультет:} Вычислительной математики и кабернетики \\
    \textbf{Кафедра:} Суперкомпьютеров и квантовой информатики \\
    \textbf{Группа:} М-218 \\
    \textbf{Студент:} Иванов Денис Евгеньевич \\
    \textbf{Научный руководитель:} к.т.н. А.С. Семенов \\
\end{minipage}
\vfill

\normalsize
В конце 2018/2019 уч. года была представлена работа, в которой были описаны две различныe идеи для
поиска черных дыр в графах. Для случая взвешенного графа предлагалось использовать муравьиный алгоритм, а для невзвешенного -- сжатие графа и топологическую сортировку.

В осеннем семестре 2019/2020 уч. года алгоритм поиска черных дыр для невзвешенного ориентированного графа с использованием топологической сортировки получил свое развитие. Было предложено разделить поиск черных дыр на две стадии: предобработка графа и поиск самих дыр.
На первом этапе мы, как и раньше, используем сжатие графа на основе конденсации компонент сильной связности.
Тем самым, задача упрощается, а также исключаются некоторые "неудобные" для полного перебора случаи.
Особенность второго этапа состоит в необходимости комбинаторого перебора. Поэтому основной задачей выбрана оптимизация этого перебора.
Для этого мы используем алгоритм, основанный на топологической структуре графа, который позволяет
значительно ускорить перебор за счет пропуска неуникальных черных дыр.

Описанный выше двухшаговый алгоритм был реализован в однопоточном режиме.
Для демонстрации его сильных и слабых сторон мы также реализовали алгоритм iBlackhole, который в 2010 году был
предложен исследователями университета New Jersey. Результаты запусков показали, что в ряде проблемных 
случаев наш алгоритм значительно превосходит iBlackhole, однако имеет и свои слабые места за счет более дорогой предобработки графа,
а также накладных расходов, связанных с вычислением необходимых данных для применения эвристики.

На данный момент алгоритм был реализован не полностью, опущена часть сложных случаев, но даже при этом продемонстрировано значительное преимущество в ряде задач. Планируется реализацию закончить и провести испытания
на полноценной версии. Есть предположение, что применимость предложенного алгоритма будет значительно зависеть от размера графа и временных ограничений
в конкретных приложениях. Также планируется реализовать более общий случай данного алгоритма и провести сравнительный анализ применимости
различных версий.

Детали работы алгоритма и проведенного исследования изложены в полной статье на английском языке, поданной на конференцию ПАВТ 2020 (см. приложение).


\end{document}
